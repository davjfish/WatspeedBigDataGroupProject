\section{Methods}

\subsection{Designing a Relational Database Schema}

The principle dataset used for the project was sourced from Montgomery County, Pennsylvania and is accessible
\href{https://www.kaggle.com/dsv/1381403}{here}~\cite{mike_chirico_2020}.
This structured, tabular dataset is provided as a flat CSV file and consists of over 600,000 records of emergency calls from December 2015 to April 2020.
This dataset will be used to create an outline of the database schema that will be used to store the dashboard data.
In order to achieve this, We will explore the dataset by importing it into a dataframe and inspect the data types being stored in each column.
Subsequently, we will decide on the best way to separate the data into discrete tables.
In addition to foreign key and uniqueness constraints, we will ensure the strategic implementation of database indexes in order to optimize database performance.
Finally, for the sake of demonstration, we decided to utilize a simple SQLite database~\cite{sqlite}.

\subsection{Dashboard REST API and Application}

In order to provide stakeholders with a concrete vision of our dashboard, our group will create a mock-up of this application.
The framework selected to do this was Django~\cite{django}---a Python object-relational mapping (ORM) web framework.
Django also has a wonderful built-in API for working with the most common types of relation database management systems.
This is great because the tables can be drafted as database-agnostic models and migrated to different systems in different environments.

Additionally, Django also has some great external packages for developing a REST API\@.
In particular, the Django REST Framework~\cite{drf} will be used to define endpoints that will be used for acquiring data from the application.
This framework provides a quick and effective way of serializing the Django models (i.e., database tables) into a JSON response as well as
writing incoming serialized data to the database.

On the frontend of the application, we will use a combination of HTML, the Django templating language and JavaScript to present end-users with
information from the dashboard.
In particular, the JavaScript library Vue.js~\cite{vue} will be important in providing users with a reactive, modern experience.
Aside from Vue.js, we will also utilize Leaflet~\cite{leaflet} and Charts.js~\cite{charts} for mapping and reactive graphing, respectively.
The mock application will be hosted in on a free-tier cloud platform in order to illicit feedback from stakeholders.

\subsection{Data Ingestion}

One of the challenges we will face in this endeavor is how to load information coming in from disparate data sources.
We will address this by creating a Python class for handling, parsing and ingesting serialized emergency call data.
A separate Parser class will have to be developed for each schema of incoming data.

\subsection{Code Repository}

All the code for this report and for the application will be stored in a publicly accessible git repository on
\href{https://github.com/davjfish/WatspeedBigDataGroupProject/}{GitHub}.


\pagebreak

