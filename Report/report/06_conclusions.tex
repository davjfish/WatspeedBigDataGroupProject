\section{Conclusions}

This project helped us develop an appreciation of the types of challenges that developers and data analysts face when
attempting to integrate large dataset from disparate sources.
In this section, we will go through each of the challenges and outline our recommendations based on this work.

\subsection{Data Acquision and Integration}

In this proof of concept, we demonstrated how a flat file, formated as tabular or serialized data (e.g., CSV or JSON), can be uploaded to a website through a
customized site administration portal.
However, in future applications, the dashboard application could have associated cron tasks that automatically connect to remote systems and
download new data.
As noted in the previous section, each source schema would have to have their custom-made parser class.

A challenge that was faced when importing data via an HTML form is that large datasets take a long time to process.
Modern-day web users do not want to wait several minutes for a form to submit.
Furthermore, large HTTP requests like this can bind up the web server.
An improvement to this system would be the introduction of asynchronousity.
For example, once the flat file was uploaded, a new asynchronous thread could be started which then calls the appropriate parser class.
Meanwhile, the end user can receive notifications about the progress (or lack thereof in the case of an error) or their import attempt.
In the context of python and Django, Celery~\cite{celery} is a great tool for the implementation of asynchronous tasks.
The use of a celery application would further require use of a messaging server such as Redis~\cite{redis}.

\subsection{Database Consideration}
- sqlite not acceptable to use in a production environment. What would we use?
this is suitable of a proof of concept however this would not be an acceptable tool to use in a production environment."
- django database routers
Horizontal Scaling use replicas etc.

Is there any room for NoSQL?

\subsection{Web Application User Permissions}

- would need classes of users and to handle permissions on the different types of views
- metadata fields such as \texttt{created\_at} and \texttt{created\_by}

\subsection{REST API Development}

it would probably make sense to put a lot of investment in the rest api to maximize interoperability between systems

