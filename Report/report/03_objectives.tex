
\section{Objectives}

\subsection{Rationale behind the analysis}

Effective emergency response is paramount to public safety, yet emergency services are often collected and monitored
 by disparate systems. Understanding trends and patterns across different jurisdictions and agencies can help
 decision-makers and public administrators make better decision about resource allocation.
 The challenge that this project is addressing is converting a high volume of raw, unstructured call data,
 such as that depicted in ~\cite{mike_chirico_2020}, into meaningful, operational intelligence.

In this project, we will discuss different approaches for data aggregation and storage and will outline some examples of ways these data can be
shared with stakeholders and administrators, alike.


\subsection{Structured Data in a SQL Database}

Storing the emergency response data in a relational database offers several critical advantages for building an effective and reliable dashboard.
 The primary rationale is centered on data integrity, efficient analysis, and compatibility with standard business intelligence tools, specifically,
 web and mobile applications that stakeholders will want to use to view and interact with the data.

There are several key reasons why a SQL database is an appropriate choice for this project.
First, most call data will have a predicable and well-structured format, e.g., incident type, timestamp, location (zip code, township), latitude and longitude.
Data without these basic attributes are going to be less useful to stakeholders.
Next, data organized into a relational database can be stored much more effectively than a two-dimensional dataframe
(assuming a well-designed schema and sufficient normalization).
A relational database management system (RDBMS) will also give administrators the ability to implement constraints in order to maximize data integrity.
This includes adding lookup tables (i.e, foreign key constraints) and enforcing uniqueness, either within a column
(e.g., should only have a single entry for a zipcode in a ZipCode table) or even between columns (e.g., combinations of 'city', 'state' and 'country' should be
unique in an AdministrativeArea table).

Finally, a SQL database is highly compatible with REST APIs, and the two technologies are commonly used together in modern application architecture.
A REST API would allow us to implement reactive tools for viewing and interacting with the data across a variety of platforms, for example, browsers, alert-systems
or mobile applications.

\subsection{Unstructured Data in a NoSQL Database}


NEED SOME HELP HERE!



\pagebreak


\begin{figure}
    \includegraphics[width=\linewidth]{Call911.jpg}
    \caption{An example image for our project.}
    \label{fig:exampleImage}
\end{figure}
