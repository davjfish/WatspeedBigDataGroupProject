
\section{Objectives}

\subsection{Rationale behind the analysis}

Effective emergency response is paramount to public safety, yet emergency services are often collected and monitored by disparate systems.
Understanding trends and patterns across different jurisdictions and agencies can help
decision-makers and public administrators make better decision about city planning, coordination, resource allocation and response optimization.
The challenge that this project is addressing is converting a high volume of raw, unstructured call data,
such as that depicted in \href{https://www.kaggle.com/dsv/1381403}{this dataset}, into meaningful, operational intelligence.

In this project, we will discuss different approaches for the aggregation, and storage of these data and will
outline some approaches for ways these data can be disseminated with stakeholders and administrators.


\subsection{Structured Data in a SQL Database}

Storing the emergency response data in a relational database offers several critical advantages for building an effective and reliable dashboard.
The primary rationale is centered on data integrity, efficient analysis, and compatibility with standard business intelligence tools, specifically,
web and mobile applications that stakeholders will want to use to view and interact with the data.

There are several key reasons why a SQL database is an appropriate choice for this project.
First, most emergency call data will have a predicable and well-structured format,
e.g., incident type, timestamp, location (zip code, township), latitude and longitude.
Data from which these basic attributes cannot be extracted are going to be less useful to stakeholders.
Next, data organized into a relational database can be stored much more effectively than a two-dimensional dataframe
(assuming a well-designed schema and sufficient normalization).
A relational database management system (RDBMS) will also give administrators the ability to implement constraints in order to maximize data integrity.
This includes adding lookup tables (i.e, foreign key constraints) and enforcing uniqueness, either within a column
(e.g., should only have a single entry for a zipcode in a ZipCode table) or between columns
(e.g., combinations of \'city\', \'state\' and \'country\' should be unique in an AdministrativeArea table).

Finally, a SQL database is highly compatible with REST APIs, and the two technologies are commonly used together in modern application architecture.
A REST API would allow for the implementation of reactive tools for viewing and interacting with the data across a variety of platforms,
such as browsers, alert-systems or mobile applications.

%
%\subsection{Unstructured Data in a NoSQL Database}
%
%NEED SOME HELP HERE!


\pagebreak

