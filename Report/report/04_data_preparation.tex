\section{Data Preparation}

\subsection{Emergency - 911 Calls from Kaggle}

The principle dataset used for the project was sourced from Montgomery County, Pennsylvania and is accessible from kaggle ~\cite{mike_chirico_2020}.
This structured, tabular dataset is provided as a flat CSV file and consists of over 600,000 records of emergency calls from December 2015 to April 2020.

After an initial inspection of the data using exploratory tools from Pandas~\cite{pandas}, a SQL schema of five tables was devised.
The five tables and their descriptions are as follows:

\begin{itemize}
    \item Category - categorical descriptions of the types of calls received (e.g., car accident)
    \item Township - township name and state (e.g., Kings Township, PA)
    \item ResponseUnit - complete list of units responding to emergency calls  (e.g., Station 123, EMS)
    \item ResponseType - the response unit type (e.g., EMS, Traffic, or Fire)
    \item EmergencyCall - this is the primary data table and used to store information about emergency calls received. It has several links to the above tables.
\end{itemize}

In addition to the foreign key constraints between tables, indexes were added to each table to improve performance.
Unique constraints were placed on fields in tables were duplication of data entry was not wanted.
For example, we only wanted there to be a single entry for \"back pain\" in the Category table and only a single entry for \"EMS\" in the ResponseType table.
The Township and ResponseUnit tables both had unique together constraints across two columns.
In the case of the former, only a single entry for a combination of township name and state was desired and for the latter,
only a single combination of response unit and station name was desired.
Finally, Figure~\ref{fig:sqldiagram} presents a visual portrayal of the five above table and the relationships between them.


\subsection{A Dashboard Application and Database}

In order to better demonstrate and a feasible use-case for this dashboard, our group created a mock-up of an application.
The framework selected for this project was Django~\cite{django}.
This python framework is a Object-Relational Mapping (ORM) tool and supports fast-paced development.
Furthermore, we were able to take advantage of the Django API to migrate changes to an actual database.
For demonstration purposes, our group decided to use a simple sqlite database however the django framework itself is \"database agnostic\".
Django also has some great packages for developing a REST API, specifically, we leveraged classes from the Django REST Framework~\cite{drf} package.

On the frontend of the application, this tool will use a combination of django templates and JavaScript (JS) libraries to present end-users with
information from the dashboard.
The JS libraries in conjunction with the REST API is a great way to provide users with a reactive experience.


The Django web-application will also be used to define a Parser class that will be used to ingest the CSV data.
Th

The mock application will be hosted in on a free-tier cloud platform in order to illicit feedback from stakeholders.

Finally, all the code for this report and for the application is store in this publicly accessible git repository on GitHub:
\url{hhttps://github.com/davjfish/WatspeedBigDataGroupProject/}.









\pagebreak

\begin{figure}
    \includegraphics[width=\linewidth]{SQL.png}
    \caption{A visual depiction of the five tables in our SQL database design.}
    \label{fig:sqldiagram}
\end{figure}


