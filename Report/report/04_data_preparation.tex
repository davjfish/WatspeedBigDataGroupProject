\section{Data Preparation}

\subsection{Designing a Relational Database Schema}

The principle dataset used for the project was sourced from Montgomery County, Pennsylvania and is accessible
\href{https://www.kaggle.com/dsv/1381403}{here}~\cite{mike_chirico_2020}.
This structured, tabular dataset is provided as a flat CSV file and consists of over 600,000 records of emergency calls from December 2015 to April 2020.
This dataset will be used to create an outline of the database schema that will be used to store the dashboard data.
In order to achieve this, We will explore the dataset by importing it into a Pandas dataframe~\cite{pandas} and inspect the types of data stored in each column.
Subsequently, we will decide on the best way to separate the data into discrete tables.
In addition to foreign key and uniqueness constraints, we will ensure the strategic implementation of database indexes in order to optimize database performance.
Finally, we will use a sqlite database---this is suitable of a proof of concept however this would not be an acceptable tool to use in a production environment.

\subsection{Dashboard REST API and Application}

In order to better demonstrate and a feasible use-case for this dashboard, our group will create a mock-up of an application.
The framework selected for this project was Django~\cite{django}---a Python object-relational mapping (ORM) framework.
Django also has a wonderful API to many commonly used SQL databases which we plan to leverage by first drafting
the tables as python classes and then migrating those classes to an actual database.

For demonstration purposes, our group decided to use a simple sqlite database however the django framework itself is \"database agnostic\".
Django also has some great packages for developing a REST API, specifically, we leveraged classes from the Django REST Framework~\cite{drf} package.

On the frontend of the application, this tool will use a combination of django templates and JavaScript (JS) libraries to present end-users with
information from the dashboard.
The JS libraries in conjunction with the REST API is a great way to provide users with a reactive experience.


The Django web-application will also be used to define a Parser class that will be used to ingest the CSV data.
Th

The mock application will be hosted in on a free-tier cloud platform in order to illicit feedback from stakeholders.

Finally, all the code for this report and for the application is store in this publicly accessible git repository on GitHub:
\url{hhttps://github.com/davjfish/WatspeedBigDataGroupProject/}.









\pagebreak

\begin{figure}
    \includegraphics[width=\linewidth]{SQL.png}
    \caption{A visual depiction of the five tables in our SQL database design.}
    \label{fig:sqldiagram}
\end{figure}


