\section{Data Preparation}

\subsection{Emergency - 911 Calls from Kaggle}

The principle dataset used for the project was sourced from Montgomery County, Pennsylvania and is accessible from kaggle ~\cite{mike_chirico_2020}.
This structured, tabular dataset is provided as a flat CSV file and consists of over 600,000 records of emergency calls from December 2015 to April 2020.

After an initial inspection of the data using exploratory tools from Pandas ~\cite{pandas}, a SQL schema of five tables was devised.
The five tables and their descriptions are as follows:

\begin{itemize}
    \item Category - categorical descriptions of the types of calls received (e.g., car accident)
    \item Township - township name and state (e.g., Kings Township, PA)
    \item ResponseUnit - complete list of units responding to emergency calls  (e.g., Station 123, EMS)
    \item ResponseType - the response unit type (e.g., EMS, Traffic, or Fire)
    \item EmergencyCall - this is the primary data table and used to store information about emergency calls received. It has several links to the above tables.
\end{itemize}

In addition to the foreign key constraints between tables, indexes were added to each table to improve performance.
Unique constraints were placed on fields in tables were duplication of data entry was not wanted.
For example, we only wanted there to be a single entry for \"back pain\" in the Category table and only a single entry for \"EMS\" in the ResponseType table.
The Township and ResponseUnit tables both had unique together constraints across two columns.
In the case of the former, only a single entry for a combination of township name and state was desired and for the latter,
only a single combination of response unit and station name was desired.




\subsection{Data quality and procurement?}

Lorem ipsum dolor sit amet, consectetur adipiscing elit, sed do eiusmod tempor incididunt ut labore et dolore magna aliqua.
Ut enim ad minim veniam, quis nostrud exercitation ullamco laboris nisi ut aliquip ex ea commodo consequat.
Duis aute irure dolor in reprehenderit in voluptate velit esse cillum dolore eu fugiat nulla pariatur.
Excepteur sint occaecat cupidatat non proident, sunt in culpa qui officia deserunt mollit anim id est laborum.

\pagebreak
